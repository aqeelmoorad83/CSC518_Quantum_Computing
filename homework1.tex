%
% Elizabeth Wilcox
% MAT 373: Number Theory
%
\documentclass[11 pt]{article}

%  packages
\usepackage{amssymb, amsmath, amsthm, enumerate, graphicx, multirow, float, color}
\usepackage{tikz}
\usepackage{hyperref}
\usetikzlibrary{arrows}

%  margins
\textwidth = 6.5 in 
\textheight = 9 in 
\oddsidemargin = 0.0 in 
\evensidemargin = 0.0 in 
\topmargin = 0.0 in 
\headheight = 0.0 in 
\headsep = 0.0 in 

%  theorems, corollaries, etc.
\theoremstyle{definition}
\newtheorem{thm}{Theorem}
\newtheorem{cor}[thm]{Corollary}
\newtheorem{lem}[thm]{Lemma}
\newtheorem{prop}[thm]{Proposition}

\theoremstyle{definition}
\newtheorem*{example}{Example}
\newtheorem{dfn}[thm]{Definition}

\renewcommand{\qedsymbol}{\vrule height 5pt width 4pt depth -1pt}


% spacing shortcuts
\def\skip{\vspace{10 pt}}
\def\medskip{\vspace{15 pt}}

% text shortcuts
\def\ie{{i.e.,}\ }
\def\eg{{e.g.,}\ }
\def\bbar#1{\overline{#1}}

%  set shortcuts
\def\QQ{\mathbb{Q}}
\def\ZZ{\mathbb{Z}}
\def\RR{\mathbb{R}}
\def\CC{\mathbb{C}}
\def\NN{\mathbb{N}}
\def\ket#1{\big|{#1}\big>}
\def\bra#1{\big<{#1}\big|}

% number theory shortcuts
\def\divides{\big |}
\def\gcd#1#2{\textrm{gcd}({#1},{#2})}
\def\ord#1{|{#1}|}

% logic shortcuts
\def\iff{\Leftrightarrow}
\thispagestyle{empty}
\begin{document}

\centerline{\bf CSC518  \hfill HW \#1}
\centerline{\bf NAME: Alexander Jansing and Brittany Zeo \hfill Mon. 21 September 2015}

\

\noindent
The kets $ \ket{+}, \ket{-}, \ket{0}, $ and $ \ket{1}$ are defined as follows:\\

\hspace*{.5cm}
\vspace*{.4cm} $\ket{0} := \left( \begin{matrix}
									1\\
									0
							   \end{matrix} \right) \text{, }
  \hspace*{.5cm} \ket{1} := \left( \begin{matrix}
									0\\
									1
							   \end{matrix} \right) \text{, }$
\hspace*{.5cm} $\ket{+} := \frac{1}{\sqrt{2}}(\ket{0} + \ket{1}) \text{, }$
\hspace*{.5cm} $\ket{-} := \frac{1}{\sqrt{2}}(\ket{0} - \ket{1})$


\vspace*{1cm}
\begin{enumerate}
\item[1] Watch the video on YouTube ( \url{https://www.youtube.com/watch?v=x6eR2vjdddY} ).

\item[2] Consider the 2-qubit state\\
$$
\ket{\psi} = \frac{1}{\sqrt{2}}\ket{0}\ket{0} + \frac{1}{\sqrt{2}}\ket{1}\ket{1}.
$$
Show that this state is entangled by proving that there are no possible values $\alpha_0, \alpha_1, \beta_0, \beta_1$ such that\\
$$
\ket{\psi} = \big(\alpha_0\ket{0} + \alpha_1\ket{1}\big)\big(\beta_0\ket{1} + \beta_1\ket{1}\big).
$$
\emph{Proof: } If there existed values $\alpha_0, \alpha_1, \beta_0, \beta_1$ such that$$
\ket{\psi} = \big(\alpha_0\ket{0} + \alpha_1\ket{1}\big)\big(\beta_0\ket{1} + \beta_1\ket{1}\big),
$$ then $$
\ket{\psi} = \alpha_0\beta_0\ket{0}\ket{0} + \alpha_0\beta_1\ket{0}\ket{1} + \alpha_1\beta_0\ket{1}\ket{0} + \alpha_1\beta_1\ket{1}\ket{1}
$$ implies that there exist some numbers such that $$\alpha_0\beta_0 = \alpha_1\beta_1 = \frac{1}{\sqrt{2}}\text{, and } \alpha_0\beta_1 = \alpha_1\beta_0 = 0.$$
\vspace*{-.4cm}$$\alpha_0\beta_1\times\beta_0 = 0\times\beta_0$$ \vspace*{-.4cm}$$\text{then }\frac{\alpha_0\beta_0\beta_1}{\beta_1} = \frac{0}{\beta_1}.$$
That implies $\alpha_0\beta_0 = \frac{1}{\sqrt{2}} = 0 \rightarrow\leftarrow$.
\begin{center}
\vspace*{-.4cm}Therefore, there are no possible values $\alpha_0, \alpha_1, \beta_0, \beta_1$ such that\vspace*{-.4cm} $$\ket{\psi} = \big(\alpha_0\ket{0} + \alpha_1\ket{1}\big)\big(\beta_0\ket{1} + \beta_1\ket{1}\big). \qed$$
\end{center}
\newpage

\item[3] Prove that $\ket{\psi}\ket{\beta_{00}} = \frac{1}{2}\ket{\beta_{00}}\ket{\psi} +\frac{1}{2}\ket{\beta_{01}}\big(X\ket{\psi}\big) +\frac{1}{2}\ket{\beta_{10}}\big(Z\ket{\psi}\big) + \frac{1}{2}\ket{\beta_{00}}\big(XZ\ket{\psi}\big).$\\
\emph{Proof: } We know that,
\begin{center}
$\begin{array}{ccccc}
&&\ket{\psi} &=& a\ket{0} + b\ket{1},\\
I &=& \ket{\beta_{00}} &=& \frac{1}{\sqrt{2}}\big(\ket{00} + \ket{11}\big),\\
X &=& \ket{\beta_{01}} &=&  \frac{1}{\sqrt{2}}\big(\ket{01} + \ket{10}\big),\\
Z &=& \ket{\beta_{10}} &=& \frac{1}{\sqrt{2}}\big(\ket{00} - \ket{11}\big),\\
XZ &=& \ket{\beta_{11}} &=&  \frac{1}{\sqrt{2}}\big(\ket{10} - \ket{01}\big).
\end{array}$
\end{center}
So,\\
$\begin{array}{ccccc}
\ket{\psi}\ket{\beta_{00}} &=& \frac{1}{2}\ket{\beta_{00}}\ket{\psi} + \frac{1}{2}\ket{\beta_{01}}\big(X\ket{\psi}\big)\\
						 &+ &\frac{1}{2}\ket{\beta_{10}}\big(Z\ket{\psi}\big) + \frac{1}{2}\ket{\beta_{00}}\big(XZ\ket{\psi}\big),\\
\\
\\
\big(a\ket{0} + b\ket{1}\big)\frac{1}{\sqrt{2}}\big(\ket{00} + \ket{11}\big) &=& \frac{1}{2}\frac{1}{\sqrt{2}}\big(\ket{00} + \ket{11}\big)\big(a\ket{0} + b\ket{1}\big)\\
						&& + \frac{1}{2}\frac{1}{\sqrt{2}}\big(\ket{01} + \ket{10}\big)\big(X\big[a\ket{0} + b\ket{1}\big]\big)\\
					    && +\frac{1}{2}\frac{1}{\sqrt{2}}\big(\ket{00} - \ket{11}\big)\big(Z\big[a\ket{0} + b\ket{1}\big]\big)\\
					    && + \frac{1}{2}\frac{1}{\sqrt{2}}\big(\ket{10} - \ket{01}\big)\big(XZ\big[a\ket{0} + b\ket{1}\big]\big).\\
\\
\\
\big(a\ket{0} + b\ket{1}\big)\big(\ket{00} + \ket{11}\big) &=& \frac{1}{2}\big[\big(\ket{00} + \ket{11}\big)\big(a\ket{0} + b\ket{1}\big)\\
						&& \big(\ket{01} + \ket{10}\big)\big(X\big[a\ket{0} + b\ket{1}\big]\big)\\
					    && \big(\ket{00} - \ket{11}\big)\big(Z\big[a\ket{0} + b\ket{1}\big]\big)\\
					    && \big(\ket{10} - \ket{01}\big)\big(XZ\big[a\ket{0} + b\ket{1}\big]\big).\\
\\
\\
a\ket{000} + b\ket{100} + a\ket{011} + b\ket{111} &=& \frac{1}{2}\big[\big(a\ket{000} + b\ket{001} + a\ket{110} + b\ket{111}\big)\\
						&&+ \big(a\ket{011} + b\ket{010} + a\ket{101} + b\ket{100}\big)\\
					    &&+ \big(a\ket{000} - b\ket{001} - a\ket{110} + b\ket{111}\big)\\
					    &&+ \big(a\ket{011} - b\ket{010} - a\ket{101} + b\ket{100}\big).\\
\\
a\ket{000} + b\ket{100} + a\ket{011} + b\ket{111} &=& \frac{1}{2}\big(2\big)\big(a\ket{000} + b\ket{100} + a\ket{011} + b\ket{111}\big)
\end{array}\\ $ \\
Therefore, $\ket{\psi}\ket{\beta_{00}} = \frac{1}{2}\ket{\beta_{00}}\ket{\psi} + \frac{1}{2}\ket{\beta_{01}}\big(X\ket{\psi}\big)
						 +\frac{1}{2}\ket{\beta_{10}}\big(Z\ket{\psi}\big) + \frac{1}{2}\ket{\beta_{00}}\big(XZ\ket{\psi}\big). \qed$
\newpage

\item[4] Which of the sets are orthonormal basis of $\mathbb{C}^2$?\\
\begin{enumerate}
\item[a] $\{\ket{+},\ket{-}\}$\\
\hspace*{1cm} $\ket{+} := \frac{1}{\sqrt{2}}(\ket{0} + \ket{1})$\\
\hspace*{1cm} $\ket{-} := \frac{1}{\sqrt{2}}(\ket{0} - \ket{1})$\\
\hspace*{1.2cm}$
\left( \begin{matrix}
\frac{1}{\sqrt{2}}\\
\frac{1}{\sqrt{2}}
\end{matrix}\right)
\cdot \left( \begin{matrix}
\frac{1}{\sqrt{2}} \\
\frac{-1}{\sqrt{2}}
\end{matrix} \right) = \frac{1}{2} - \frac{1}{2} = 0\\
\perp\in\mathbb{C}$\\

\item[b] $\{\ket{0},\ket{1}\}$\\
\hspace*{1cm} $\ket{0} = \left( \begin{matrix}
1\\0
\end{matrix}\right)$\\
\hspace*{1cm} $\ket{1} = \left( \begin{matrix}
0\\1
\end{matrix}\right)$\\
\hspace*{1.2cm} $\left( \begin{matrix}
1\\0
\end{matrix}\right)\cdot\left( \begin{matrix}
0\\1
\end{matrix}\right) = 0\\
\perp\in\mathbb{C}$\\

\item[c] $\{\ket{0} - \ket{1},\ket{1} + \ket{0}\}$\\
\hspace*{1cm} $\ket{0} - \ket{1} = \left( \begin{matrix}
1\\-1
\end{matrix}\right)$\\
\hspace*{1cm} $\ket{1} + \ket{0} = \left( \begin{matrix}
1\\1
\end{matrix}\right)$\\
\hspace*{1.2cm}$\left( \begin{matrix}
1\\-1
\end{matrix}\right)\cdot\left( \begin{matrix}
1\\1
\end{matrix}\right) = 1 - 1 = 0\\
\perp\in\mathbb{C}$\\

\item[d] $\{\frac{1}{\sqrt{2}}\left( \begin{matrix} 1\\i \end{matrix} \right), \frac{1}{\sqrt{2}} \left( \begin{matrix}1\\-i \end{matrix} \right) \}$\\
\hspace*{1.2cm}$\frac{1}{\sqrt{2}}\big( \left( \begin{matrix} 1\\i\end{matrix}\right)\cdot\ \left( \begin{matrix}1\\-i\end{matrix} \right) \big) = \frac{2}{\sqrt{2}} + \frac{i-i}{\sqrt{2}} = \frac{2}{\sqrt{2}}\\
\textbf{NOT}\perp\in\mathbb{C}$\\

\item[e] $\{\frac{1}{\sqrt{2}}\left(\begin{matrix}\cos\theta \\ \sin\theta\end{matrix} \right), \frac{1}{\sqrt{2}} \left( \begin{matrix}\sin\theta \\ -\cos\theta \end{matrix} \right)\}$\\
\hspace*{1.2cm} $\frac{1}{\sqrt{2}}\big(\left( \begin{matrix}\cos\theta \\ \sin\theta \end{matrix} \right)\cdot\left( \begin{matrix} \sin\theta \\ -\cos\theta \end{matrix} \right)\big) = \frac{1}{\sqrt{2}}\big( \cos\theta\sin\theta - \cos\theta\sin\theta\big) = 0\\
\perp\in\mathbb{C}$\\

\item[f] $\{\frac{1}{\sqrt{2}}\left(\begin{matrix}1\\i\end{matrix}\right) ,\frac{1}{\sqrt{2}}\left(\begin{matrix}1\\-1\end{matrix}\right)\}$\\
\hspace*{1.2cm} $\frac{1}{\sqrt{2}}\big(\left(\begin{matrix}1\\i\end{matrix}\right)\cdot\left(\begin{matrix}1\\-1\end{matrix}\right)\big) = \frac{1}{\sqrt{2}} \big( 1 - i \big) \neq 0\\ \textbf{NOT}\perp\in\mathbb{C}$\\
\end{enumerate}
\newpage
\item[5] Let $\ket{\psi},\ket{\varphi}$ be an orthonormal basis in the Hilbert space $\mathbb{C}^2$.\\
Let $A := \ket{\psi}\bra{\psi} + \ket{\varphi}\bra{\varphi}$.\\
\vspace*{1cm}
Find the matrix representation of $A$ with respect to the basis $\{\ket{0}, \ket{1}\}$ where $\ket{\psi}$ and $\ket{\varphi}$ are as follows:\\
\begin{enumerate}
\item[a] $\ket{\psi} := \left(\begin{matrix}1\\0\end{matrix}\right), \ket{\varphi} := \left( \begin{matrix}0\\1\end{matrix}\right)$\\
\hspace*{1.2cm} $\left( \begin{array}{cc}1&0\\0&0\end{array}\right) + \left( \begin{array}{cc}0&0\\0&1\end{array}\right) = \left( \begin{array}{cc}1&0\\0&1\end{array}\right)$\\

\item[b] $\ket{\psi} := \frac{1}{\sqrt{2}}\left( \begin{array}{c}1\\1\end{array}\right), \ket{\varphi} := \frac{1}{\sqrt{2}}\left( \begin{array}{c}1\\-1\end{array}\right)$\\
\hspace*{1.2cm}$\left( \begin{array}{cc}\frac{1}{2}&\frac{1}{2}\\\frac{1}{2}&\frac{1}{2}\end{array}\right)+\left( \begin{array}{cc}\frac{1}{2}&\frac{-1}{2}\\\frac{-1}{2}&\frac{1}{2}\end{array}\right) = \left( \begin{array}{cc}1&0\\0&1\end{array}\right)$\\
\item[c] $\ket{\psi} := \left( \begin{array}{c}\cos\theta\\\sin\theta\end{array}\right), \ket{\varphi} := \left( \begin{array}{c}\sin\theta\\-\cos\theta\end{array}\right)$\\
\hspace*{1.2cm}$\left( \begin{array}{cc}\cos^2\theta & \cos\theta\sin\theta\\
									\cos\theta\sin\theta & \sin^2\theta\end{array}\right) + 
			   \left( \begin{array}{cc}\sin^2\theta & -\cos\theta\sin\theta\\
									-\cos\theta\sin\theta & \cos^2\theta\end{array}\right) =\\ 
			   \hspace*{6cm}\left( \begin{array}{cc}\cos^2\theta + \sin^2\theta & 0\\
									0 & \sin^2\theta + \cos^2\theta\end{array}\right) = \left( \begin{array}{cc}1&0 \\ 0&1\end{array}\right)  $
\end{enumerate}
\newpage
\item[6] $T = \ket{0}\bra{0}\otimes\big(\ket{+}\bra{+} + \ket{-}\bra{-}\big) + \ket{1}\bra{1}\otimes\big(\ket{+}\bra{+} - \ket{-}\bra{-}\big)$.\\
\vspace*{.5cm}
Show that $T$ and $CNOT$ are equal.\\
\vspace*{.5cm}$T = \left( \begin{array}{cc}1&0 \\ 0&0\end{array}\right)\otimes\frac{1}{2}\left( \begin{array}{cc}2&0 \\ 0&2\end{array}\right) + \left( \begin{array}{cc}0&0 \\ 0&1\end{array}\right)\otimes\frac{1}{2}\left( \begin{array}{cc}0&2 \\ 2&0\end{array}\right).\\
T = \left( \begin{array}{cc}1&0 \\ 0&0\end{array}\right)\otimes\left( \begin{array}{cc}1&0 \\ 0&1\end{array}\right) + \left( \begin{array}{cc}0&0 \\ 0&1\end{array}\right)\otimes\left( \begin{array}{cc}0&1 \\ 1&0\end{array}\right)\\
T = \left( \begin{array}{cccc}1 & 0 & 0 & 0\\
						    0 & 1 & 0 & 0\\
						    0 & 0 & 0 & 1\\
						    0 & 0 & 1 & 0\\ \end{array}\right) = CNOT
$\\
\item[7] Compute:
\begin{enumerate}
\item[a] $\bra{10}\ket{+-}$ and $\bra{1}\ket{+}\bra{0}\ket{-}$\\
$\bra{10}\ket{+-} = \left( \begin{array}{cccc}0&0&1&0\end{array}\right)\cdot\left( \begin{array}{c}\frac{1}{2}\\\frac{-1}{2}\\\frac{1}{2}\\\frac{-1}{2}\end{array}\right) = \frac{1}{2}$\\
$\bra{1}\ket{+}\bra{0}\ket{-} = \left( \begin{array}{cc}0&1\end{array}\right)\cdot\left( \begin{array}{c}\frac{1}{\sqrt{2}}\\\frac{1}{\sqrt{2}}\end{array}\right) \otimes \left( \begin{array}{cc}1&0\end{array}\right)\cdot\left( \begin{array}{c}\frac{1}{\sqrt{2}}\\\frac{-1}{\sqrt{2}}\end{array}\right) = \frac{1}{\sqrt{2}}\otimes\frac{1}{\sqrt{2}} = \frac{1}{2}$\\

\item[b] $\bra{01}\ket{-+}$ and $\bra{0}\ket{-}\bra{1}\ket{+}$\\
$\bra{01}\ket{-+} = \left( \begin{array}{cccc}0&1&0&0\end{array}\right)\cdot\left( \begin{array}{c}\frac{1}{2}\\\frac{1}{2}\\\frac{-1}{2}\\\frac{-1}{2}\end{array}\right) = \frac{1}{2}$\\
$\bra{0}\ket{-}\bra{1}\ket{+} = \left( \begin{array}{cc}1&0\end{array}\right)\cdot\left( \begin{array}{c}\frac{1}{\sqrt{2}}\\\frac{-1}{\sqrt{2}}\end{array}\right) \otimes \left( \begin{array}{cc}0&1\end{array}\right)\cdot\left( \begin{array}{c}\frac{1}{\sqrt{2}}\\\frac{1}{\sqrt{2}}\end{array}\right)  = \frac{1}{\sqrt{2}}\otimes\frac{1}{\sqrt{2}} = \frac{1}{2}$\\

\item[c] Let $\ket{\psi} = \ket{\psi_1} \otimes \ket{\psi_2}$ and $\ket{\phi} = \ket{\phi_1} \otimes \ket{\phi_2}$, where $\ket{\psi_1}, \ket{\psi_2}, \ket{\phi_1}, \ket{\phi_2}$ are arbitrary vectors in $\mathbb{C}^2$. Show that $\bra{\psi}\ket{\phi} = \bra{\psi_1}\ket{\phi_1}\bra{\psi_2}\ket{\phi_2}$.\\
For arbitrary values $a, b, c, d, e, f, g, $ and $h \in \mathbb{C}$,\\
$\ket{\psi_1} = a\ket{0} + b\ket{1} = \left(\begin{array}{c}a\\b\end{array}\right)\\
\ket{\psi_2} = c\ket{0} + d\ket{1} = \left(\begin{array}{c}c\\d\end{array}\right)\\
\ket{\phi_1} = e\ket{0} + f\ket{1} = \left(\begin{array}{c}e\\f\end{array}\right)\\
\ket{\phi_2} = g\ket{0} + h\ket{1} = \left(\begin{array}{c}g\\h\end{array}\right)\\
\ket{\psi} = \left(\begin{array}{c}ac \\ ad \\ bc \\bd\end{array}\right), 
\ket{\phi} = \left(\begin{array}{c}eg \\ eh \\ fg \\fh\end{array}\right)\\
\bra{\psi}\ket{\phi} = aceg + adeh + bcfg + bdfh\\ 
\bra{\psi_1}\ket{\phi_1}\bra{\psi_2}\ket{\phi_2} = \big(ae + bf\big)\otimes\big(cg + dh\big) = aecg + aedh + bfcg + bfdh.
$
\end{enumerate}
\end{enumerate}



\end{document}