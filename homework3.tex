%%%%%%%%%%%%%%%%%%%%%%%%%%%%%%%%%%%%%%%%%%%%%%%
%%%This is a science homework template. Modify the preamble to suit your needs.
%%%%%%%%%%%%%%%%%%%%%%%%%%%%%%%%%%%%%%%%%%%%%%%%

\documentclass[12pt]{article}

\usepackage{amssymb,amsmath,amsthm}
\usepackage[margin=1.25in]{geometry}
\usepackage{graphicx,ctable,booktabs, tikz}
\usepackage{hyperref}
\usepackage{listings} %for inserting code snippets
%%%%%%%%%%%%%%%%%Quantum Circuist%%%%%%%%%%%%%%%%%%%%%%
\usepackage{etex}
\usepackage[all]{xy}
\SelectTips{em}{}
\vfuzz2pt % Don't report over-full v-boxes if over-edge is small
%\usepackage[all]{xy}
%%%%%%%%%%%%%%%%%%%%%%%%%%%%%%%%%%%%%%%%%%%%%%%%%%%%%%%
\usepackage{multicol}

\newcommand{\<}{\langle}
\renewcommand{\>}{\rangle}
\newcommand{\C}{\mathbb{C}}
\newcommand{\cA}{\mathcal{A}}
\newcommand{\cB}{\mathcal{B}}
\newcommand{\cC}{\mathcal{C}}
\newcommand{\cE}{\mathcal{E}}
\newcommand{\cF}{\mathcal{F}}
\newcommand{\cH}{\mathcal{H}}
\newcommand{\cK}{\mathcal{K}}
\newcommand{\cM}{\mathcal{M}}
\newcommand{\cR}{\mathcal{R}}
\newcommand{\cU}{\mathcal{U}}
\newcommand{\cV}{\mathcal{V}}
\newcommand{\cZ}{\mathcal{Z}}
\newcommand{\E}{\mathrm{\mathbf{E}}}
\newcommand{\Var}{\mathrm{\mathbf{Var}}}
\newcommand{\EE}[1]{\E\left(#1\right)}
\newcommand{\VV}[1]{\Var\left(#1\right)}
\newcommand{\s}{\mathrm{span}}
\newcommand{\R}{\mathbb{R}}
\renewcommand{\sim}{\mathrm{sim}}
\renewcommand{\prec}{\mathrm{prec}}
\newcommand{\fail}{\mathrm{fail}}
\newcommand{\median}{\mathrm{median}}
\newcommand{\anc}{\mathrm{anc}}

\renewcommand{\labelitemii}{$\circ$}

% \newcommand{\ket}[1]{\left| #1\right\rangle}      % ket vector
% \newcommand{\bra}[1]{\left\langle #1\right|}      % bra vector
\newcommand{\kets}[1]{| #1 \rangle}                 % small ket vector
\newcommand{\bras}[1]{\langle #1 |}                 % small bra vector
\newcommand{\braket}[2]{\langle #1 | #2 \rangle}         % <x|y>
\newcommand{\ii}{\mathbb{I}}

% the I with two vertical lines
\newcommand{\twonorm}[1]{\left\| #1\right\|_2}        % norm
\newcommand{\inftynorm}[1]{\left\| #1\right\|_\infty}
\newcommand{\norms}[1]{\big\| #1\big\|}          % norm

\newcommand{\tr}{\mathrm{tr}}
%\newcommand{\qed}{\hfill $\square$}
\newtheorem{definition}{Definition}
\newtheorem{theorem}{Theorem}
\newtheorem{lemma}{Lemma}
\newtheorem{proposition}{Proposition}
\newtheorem{cor}{Corollary}
\newtheorem{remark}{Remark}


\makeatletter
\newenvironment{problem}{\@startsection
       {section}
       {1}
       {-.2em}
       {-3.5ex plus -1ex minus -.2ex}
       {2.3ex plus .2ex}
       {\pagebreak[3]
       \large\bf\noindent{Problem }
       }
       }
       {%\vspace{1ex}\begin{center} \rule{0.3\linewidth}{.3pt}\end{center}}
       \begin{center}\large\bf \end{center}}
\makeatother


%
%Fancy-header package to modify header/page numbering
%
\usepackage{fancyhdr}
\pagestyle{fancy}
%\addtolength{\headwidth}{\marginparsep} %these change header-rule width
%\addtolength{\headwidth}{\marginparwidth}
\lhead{Problem \thesection}
\chead{}
\rhead{\thepage}
\lfoot{\small\scshape CS 518}
\cfoot{}
\rfoot{\footnotesize }
\renewcommand{\headrulewidth}{.3pt}
\renewcommand{\footrulewidth}{.3pt}
\setlength\voffset{-0.25in}
\setlength\textheight{648pt}

%%%%%%%%%%%%%%%%%%%%%%%%%%%%%%%%%%%%%%%%%%%%%%%
\def\QQ{\mathbb{Q}}
\def\ZZ{\mathbb{Z}}
\def\RR{\mathbb{R}}
\def\CC{\mathbb{C}}
\def\NN{\mathbb{N}}
\def\ket#1{\big|{#1}\big>}
\def\bra#1{\big<{#1}\big|}
\def\braket#1#2{\big<{#1}\big|{#2}\big>}
\def\ketbra#1{\big|{#1}\big>\big<{#1}\big|}
\def\sqtwo{\sqrt{2}}
\pagestyle{empty}
\begin{document}
\centerline{\bf CSC518  \hfill HW \#3}
\centerline{\bf NAME: Alexander Jansing and Brittany Zeo \hfill \today}

\

\noindent

\begin{problem}{Simon's Algorithm: 28 pts}
Suppose we run Simon's algorithm on the following input $x$ (with $N = 8$ and hence $n = 3$):  We have
$$\begin{array}{cc}
x_{000} = x_{111} = 000 & x_{001} = x_{110} = 001\\
x_{010} = x_{101} = 010 & x_{011} = x_{100} = 011
\end{array}
$$
Note that $x$ is $2$-to-$1$ and $x_i = x_{i\oplus 111}$ for all $i \in \{0,1\}^3$, so $s = 111$.
\begin{enumerate}
\item[a] Give the starting state of Simon's algorithm.\\
$$\begin{array}{ccc}
U_f\ket{x}\ket{0}^{\otimes n} &=& \ket{x}\ket{0\oplus f(x)}\\
\otimes_{l = 0}^n\ket{0} &=& \otimes_{l = 0}^3\ket{0}
\end{array} $$
\item[b] Give the state after the first Hadamard transforms on the first 3 qubits.\\
$$\begin{array}{ccc}
H^{\otimes n}\ket{x} &=& \frac{1}{\sqrt{2^n}}\sum\limits_{z\in\{0,1\}^n}(-1)^{x\cdot z}\ket{z},\\
H^{\otimes 3}\ket{x} &=& \frac{1}{\sqrt{8}}\sum\limits_{z\in\{0,1\}^3}(-1)^{x\cdot z}\ket{z} \otimes \ket{0}.
\end{array} $$
\item[c] Give the state after applying the oracle.\\
$$\begin{array}{ccc}
\frac{1}{\sqrt{2^{n-1}}}\sum\limits_{x\in 1}\frac{1}{\sqrt{2}}(\ket{x} + \ket{x\oplus s})\ket{f(x)}\\
\frac{1}{2}\sum\limits_{x\in 1}\frac{1}{\sqrt{2}}(\ket{x} + \ket{x\oplus s})\ket{f(x)}\\
\end{array}$$
\item[d] Give the state after measuring the second register (the measurement gave $\ket{001}$).\\
Measurement collapses the state and since the measurement gave $\ket{001}$, it is also the state; normalizing to $1$. 
\item[e] Use $H^{\otimes n}\ket{i} = \frac{1}{\sqrt{2^n}}\sum\limits_{j\in\{0,1\}^n}(-1)^{i\cdot j}\ket{j}$ to give the state after the final Hadamard.\\
$H^{\otimes n}\ket{i} = \frac{1}{\sqrt{2^n}}\sum\limits_{j\in\{0,1\}^n}(-1)^{i\cdot j}\ket{j}$\\
$H^{\otimes n}(\frac{1}{\sqrt{2}}\left[\ket{000...0}+\ket{s}\right]) = \frac{1}{\sqrt{2^{n+1}}}\sum\limits_{j\in\{0,1\}^n}(2)\ket{j}$\\
$\frac{1}{\sqrt{2^{n})}}\sum\limits_{j\in s^\perp}\ket{j}$\\

\item[f] Why does measurement of the first $3$ qubits of the final state give the information about $s$?\\
Because the bits in the first register are in a superpositioned state and as they pass through $U_f$ they output the state that is the \texttt{XOR} of the second register and $s$.
\item[g] Suppose the first run of the algorithm give $j = 011$ and the second run gives $j = 101$. Show that, assuming $s \neq 000$, those two runs already determine $s$.\\
\end{enumerate}
\end{problem}

\begin{problem}{Fourier Transform: 30 pts}
\begin{enumerate}
\item[a] For $\omega = e^{\frac{2\pi i}{3}}$ and $F_3  = \frac{1}{\sqrt{3}}\left(\begin{array}{ccc}
																      1&1&1\\
																      1&\omega&\omega^2\\
																      1&\omega^2&\omega\\
																	  \end{array}  \right)$
, calculated $F_3\left(\begin{array}{c}
0\\1\\0
\end{array}  \right)$ and  $F_3\left(\begin{array}{c}
1\\\omega^2\\\omega
\end{array}  \right)$.\\
\begin{enumerate}
\item[$\cdot$] $F_3\left(\begin{array}{c}
0\\1\\0
\end{array}  \right) = \frac{1}{\sqrt{3}}\left(\begin{array}{c}1\\\omega^2\\\omega^2\end{array}\right)$
\item[$\cdot$] $F_3\left(\begin{array}{c}
1\\\omega^2\\\omega
\end{array}  \right) = \frac{1}{\sqrt{3}}\left(\begin{array}{c}\omega + \omega^2 + 1 \\ 2\omega^3 + 1 \\ \omega^2 + \omega^4 + 1\end{array}\right)$
\end{enumerate}
\item[b] Let the Fourier transform defined as what we described in class, \emph{i.e.} $\omega = e^{\frac{2\pi i}{N}}$ and entry at location $(i,j)$ is $e^{\frac{2\pi ijk}{N}}$ where $0 \leq j,k < N$. Let $\ket{C_k}$ be the $k^{th}$ column of $F_N$. Please show that
$$\braket{C_k'}{C_k} = \begin{cases}
						1 \text{ if } k = k'\\
						0 \text{ if } k \neq k'\\
					  \end{cases}$$
A Fourier transform square matrix of size $N$ can be seen as such,\\
$$\frac{1}{\sqrt{N}}\left(\begin{array}{ccccccc}
1 & 1 & 1 & \cdots & \cdots & \cdots & 1\\
1 & \omega & \omega^{1*2} &\ddots&&& \omega^{1*(N-1)}\\
1 & \omega^{2*1} & \omega^{2*2} &&\ddots&& \omega^{2*(N-1)}\\
\vdots & \vdots & \vdots &&&\ddots& \vdots\\
1 & \omega^{(N-1)*1} &&&\cdots&& \omega^{(N-1)*(N-1)}
\end{array}\right)$$
We know that QFT matrices are \emph{unitary} and therefore the column vectors are linearly independent. Therefore each column vector is ``parallel" to itself, and its $\braket{C_k'}{C_k} = 1$, and is orthogonal to the other column vectors in $N$-space, so $\braket{C_k'}{C_k} = 0$.
%That is to say, when we find the inner product between columns of a Fourier transform square matrix, we're looking at
%$$\frac{1}{\sqrt{N}}\left(\begin{array}{cccc}
%1& \overline{\omega^{1*k'}} & \cdots & \overline{\omega^{(N-1)*k'}}
%\end{array}\right) \cdot \frac{1}{\sqrt{N}}\left(\begin{array}{c}
%\omega^{0*k}\\
%\omega^{1*k}\\
%\vdots\\
%\omega^{(N-1)*k}
%\end{array}\right)$$ $$ = \frac{1}{N}(1+ \overline{\omega^{k}}\omega{k} + \cdots + \overline{\omega^{(N-1)*k'}}\omega^{(N-1)*k}).$$
%If $k \neq k'$, then we have
%$$\frac{1}{N}(1+ \omega^{k' + k} + \cdots + \omega^{(N-1)*(k' +k)}).$$
%And if $k = k'$, then we have
%$$\frac{1}{N}(1+ \omega^{2k} + \cdots + \omega^{2(N-1)*k}).$$
%Then to each other these, we can expand:
%\begin{enumerate}
%\item[$k\neq k':$] $\frac{1}{N}+ \frac{e^{\frac{2\pi i(k' + k)}{N}}}{N} + \cdots + \frac{e^{\frac{2\pi i(N-1)*(k' +k)}{N}}}{N} = 0.$\\
%                   $\left(\frac{1}{N}+ \frac{e^{\frac{2\pi i(k' + k)}{N}}}{N} + \cdots + \frac{e^{\frac{(2\pi i(N-1)*(k' +k)}{N}}}{N}\right)*N = 0*N.$\\
%                   $\left(1+ e^{\frac{2\pi i(k' + k)}{N}} + \cdots + e^{\frac{2\pi i(N-1)*(k' +k)}{N}}\right)*e^{\frac{2\pi i}{N}} = 0*e^{\frac{2\pi i}{N}}$\\
%                   $\left((e^{\frac{2\pi i}{N}}*e^{\frac{2\pi i}{N}})+ e^{\frac{2\pi i(k' + k) + 2\pi i}{N}} + \cdots + e^{\frac{2\pi i(N-1)*(k' +k) + 2\pi i}{N}}\right) = 0*e^{\frac{2\pi i}{N}}$\\
%                   $\left(1+ e^{\frac{2\pi i(k' + k) + 2\pi i}{N}} + \cdots + e^{\frac{2\pi i(N)*(k' +k)}{N}}\right) = 0$\\
%                   $\left(1+ e^{\frac{2\pi i(k' + k)}{N}} + \cdots + e^{2\pi i*(k' +k)}\right) = 0$\\
%
%\end{enumerate}
\newpage
\item[c] Prove the identity in equation 7.1.18 in the textbook.\\
Identity 7.1.18 states, for some binary $w = 0.x_1x_2\cdots$\\
\textbf{Is it a typo in the book when it says, ``$\sum\limits_{y=0}^{2^n-1}$?" Shouldn't it be $\sum\limits_{y=0}^{2^{n-1}}$?}\\
\hspace*{2cm} $\frac{1}{\sqrt{2^n}}\sum\limits_{y=0}^{2^{n-1}}e^{2\pi i wy}\ket{y} = \left( \frac{\ket{0} + e^{2\pi i(2^{n-1}w)}\ket{1}}{\sqrt{2}} \right) \otimes \left( \frac{\ket{0} + e^{2\pi i(2^{n-2}w)}\ket{1}}{\sqrt{2}} \right) \otimes \cdots\\ \hspace*{11cm} \cdots \otimes \left( \frac{\ket{0} + e^{2\pi i(w)}\ket{1}}{\sqrt{2}} \right)$.\\
So, we know in a general case from the notes,\\
$\begin{array}{ccccccc}
\mu^{2^0}\ket{\psi} &=& e^{2\pi i(w)} &=& e^{2\pi i(0.x_1x_2\cdots)} &=& \left( \frac{\ket{0} + e^{2\pi i(0.x_1x_2\cdots)}\ket{1}}{\sqrt{2}} \right)\\
\mu^{2^1}\ket{\psi} &=& e^{2\pi i(2^1w)} &=& e^{2\pi i(x_1.x_2x_3\cdots)} &=& \left( \frac{\ket{0} + e^{2\pi i(0.x_2x_3\cdots)}\ket{1}}{\sqrt{2}} \right)\\
&&&\vdots&&&\\
\mu^{2^{n-1}}\ket{\psi} &=& e^{2\pi i(2^{n-1}w)} &=& e^{2\pi i(x_1\cdots x_{n-1}.x_n)} &=& \left( \frac{\ket{0} + e^{2\pi i(0.x_n)}\ket{1}}{\sqrt{2}} \right)\\
\end{array}$\\
So $\sum\limits_{y=0}^{2^{n-1}}e^{2\pi i wy}\ket{y} = \left( \frac{\ket{0} + e^{2\pi i(2^{n-1}w)}\ket{1}}{\sqrt{2}} \right) \otimes \left( \frac{\ket{0} + e^{2\pi i(2^{n-2}w)}\ket{1}}{\sqrt{2}} \right) \otimes \cdots \otimes \left( \frac{\ket{0} + e^{2\pi i(w)}\ket{1}}{\sqrt{2}} \right)$ easily follows.
\end{enumerate}
\end{problem}

\begin{problem}{Euclidean distance: 12 pts}
\emph{The Euclidean distance} between two states $\ket{\phi} = \sum_i\alpha_i\ket{i}$ and $\ket{\psi} = \sum_i\beta_i\ket{i}$ and $\left|\left|\ket{\phi}-\ket{\psi}\right|\right| = \sqrt{\sum_i\left| \alpha_i - \beta_i\right|^2}$. Assume the states are unit vectors with real amplitudes. Suppose the distance is small: $\left|\left|\ket{\phi}-\ket{\psi}\right|\right| = \epsilon$. Show that the probabilities resulting from a measurement on the two states are also close: $\sqrt{\sum_i\left| \alpha_i^2 - \beta_i^2\right|} \leq 2\epsilon$ (Hint: use \emph{Cauchy-Schwarz} inequality).\\
$$\begin{array}{ccc}
\sqrt{\sum_i|\alpha_i - \beta_i|^2} &\leq & \sqrt{\sum_i|\alpha_i^2 - \beta_i^2|^2}\\
\sqrt{\sum_i|\alpha_i^2 - \beta_i^2|^2} &=& \sqrt{\sum_i\left(|\alpha_i - \beta_i|*|\alpha_i + \beta_i|\right)^2}\\
\sqrt{\sum_i\left(|\alpha_i - \beta_i|*|\alpha_i + \beta_i|\right)^2} & \leq & \sqrt{\sum_i|\alpha_i - \beta_i|^2*\sum_i|\alpha_i + \beta_i|^2}\\
\cdots & & \sqrt{\epsilon^2*\sum_i|\alpha_i + \beta_i|^2}\\
\sqrt{\epsilon^2*\sum_i|\alpha_i + \beta_i|^2} &=& \epsilon\sqrt{\sum_i|\alpha_i + \beta_i|^2}\\
\epsilon\sqrt{\sum_i|\alpha_i + \beta_i|^2} &\leq & \epsilon*\left(\sum_i|\alpha_i|^2 + \sum_i|\beta_i|^2\right)\\
\epsilon*\left(\sum_i|\alpha_i|^2 + \sum_i|\beta_i|^2\right)  &=& \epsilon*2
\end{array} $$
\end{problem}

\begin{problem}{Analysis Technique Proof: 10 pts}
In quantum counting or the hard case analysis of Shor’s algorithm, the following
analysis technique is commonly used: $\left| 1 - e^{i\theta} \right| = 2\left| \sin (\frac{\theta}{2})\right|$ Please prove this equality.\\
\begin{center}
$\begin{array}{cccc}
\left| 1 - e^{i\theta} \right| &=& 2\left| \sin (\frac{\theta}{2})\right|\\
\left| 1 - (\cos (\theta) + i\sin (\theta)) \right| &=& 2\left| \sin (\frac{\theta}{2})\right|\\
\left| 1 - \cos (\theta) - i\sin (\theta)) \right| &=& 2\left| \sin (\frac{\theta}{2})\right|\\
\left| 1 - \cos \left( \frac{2\theta}{2} \right) - i\sin \left(\frac{2\theta}{2} \right) \right| &=& 2\left| \sin (\frac{\theta}{2})\right|\\
\left| 1 - 1 + 2\sin^2 \left( \frac{\theta}{2} \right) - 2i\cos \left(\frac{\theta}{2} \right)\sin \left(\frac{\theta}{2}\right) \right| &=& 2\left| \sin (\frac{\theta}{2})\right|\\
2 \left|\sin \left( \frac{\theta}{2} \right) \right| \left| \sin \left( \frac{\theta}{2} \right) - i\cos \left( \frac{\theta}{2} \right) \right| &=& 2\left| \sin (\frac{\theta}{2})\right|\\
2 \left|\sin \left( \frac{\theta}{2} \right) \right| \left| \cos \left( \frac{\pi}{2} - \theta \right) - i\sin \left( \frac{\pi}{2} - \theta \right) \right| &=& 2\left| \sin (\frac{\theta}{2})\right|\\
2 \left|\sin \left( \frac{\theta}{2} \right) \right| \left| e^{-i\left( \frac{\pi}{2} - \theta\right)} \right| &=& 2\left| \sin (\frac{\theta}{2})\right|\\
2 \left|\sin \left( \frac{\theta}{2} \right) \right| 1 &=& 2\left| \sin (\frac{\theta}{2})\right|\\
2 \left|\sin \left( \frac{\theta}{2} \right) \right| &=& 2\left| \sin (\frac{\theta}{2})\right|&\checkmark
\end{array}$
\end{center}
\end{problem}

\begin{problem}{Root of Unity: 10 pts}
Prove that if a operator $U$ satisfies $U^r = I$, then the eigenvalues of U must be $r^{th}$ roots of $1$.\\
If $U$ was unitary, then $U^1\ket{\psi} = \lambda\ket{\psi}$ and since $U$ would be unitary $\lambda$ (the eigenvalue(s) of similar $1's$ we're looking for) would have a magnitude of $1$, so $\lambda = e^{2\pi i\phi}$ where $\phi \in [0,1)$.\\
\begin{center}
$\vdots$
\end{center}
Now if $U$ is $r$-nary, then $U^r\ket{\psi} = \lambda\ket{\psi}$ and since $U$ would be $r$-nary $\lambda$  would have a magnitude of $r$, so $\lambda = e^{2\pi i\phi}$ where $\phi \in [0,r)$; for some $r\in\ZZ$ and $\phi\in\RR$.\\
\end{problem}
\newpage

\begin{problem}{Gate Approximation: 10 pts}
As mentioned in class that the implementation of QFT inverse will be difficult if the precision requirement is high for the control rotation gates. Based on the Solovay-Kitaev’s decomposition theorem, there is always a way to approximate a single qubit gate with error at most using $O(log^c\frac{1}{\epsilon})$ gates from the universal gate where the optimal $c$ is some number slightly less than $2$. Please describe (sketch) the proof of this theorem.\\
This we're both unsure of how to begin without just copying the material from here within the time we have; and already being late with this. So I will reference material from \href{http://home.lu.lv/~sd20008/papers/essays/Solovay-Kitaev.pdf}{here} and admit that I'd look here for at someone who has gone through the work to prove the theorem.\\

URL for paper version: http://home.lu.lv/~sd20008/papers/essays/Solovay-Kitaev.pdf
\end{problem}


\end{document}